\begin{table}[htbp]\centering \def\sym#1{\ifmmode^{#1}\else\(^{#1}\)\fi} \caption{Summary statistics by attrition} \label{tab:sum-attr1} \maxsizebox*{\textwidth}{\textheight}{ \begin{threeparttable} \begin{tabular}{l*{3}{c}} \toprule
          &\multicolumn{3}{c}{Mean (SD)}\\\cmidrule(lr){2-4}
          &\multicolumn{1}{c}{Complete}&\multicolumn{1}{c}{Attrition}&\multicolumn{1}{c}{\specialcell{Complete -\\Attrition}}\\
\midrule
Female    &     0.58&     0.59&     0.88\\
          &(0.49) 284 &(0.50) 27 &         \\
Age       &    31.39&    29.78&     0.41\\
          &(9.79) 276 &(8.36) 27 &         \\
Completed std. 8&     0.98&     0.93&0.06$^{*}$\\
          &(0.13) 284 &(0.27) 27 &         \\
Married/co-habitating&     0.49&     0.44&     0.66\\
          &(0.50) 280 &(0.51) 27 &         \\
No. of children&     1.91&     1.85&     0.86\\
          &(1.75) 284 &(1.83) 27 &         \\
Constant relative risk aversion&     1.18&     1.19&     0.98\\
          &(1.30) 284 &(1.30) 27 &         \\
Locus of control&    69.70&    69.63&     0.97\\
          &(10.38) 284 &(7.71) 27 &         \\
\bottomrule \end{tabular} \begin{tablenotes}[flushleft] \footnotesize \item \emph{Notes:} The first two columns report means of each row variable by observation status at endline. SD are in parentheses with sample size. The last column report the \emph{p}-value for a difference of means \emph{t}-test between each group. * denotes significance at 10 pct., ** at 5 pct., and *** at 1 pct. level. \end{tablenotes} \end{threeparttable} } \end{table}

% File produced by sum-participation.do with /n/homeserver2/user2a/justinra/repos/akiba-lottery-pub/data/clean/akiba_wide.dta on 02:02:57 18 Apr 2017 by user justinra on Stata 13.1 with seed X53d8cd0fc43f462544a474abacbdd93d00044a8f