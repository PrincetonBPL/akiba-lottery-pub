\begin{table}[htbp]\centering \def\sym#1{\ifmmode^{#1}\else\(^{#1}\)\fi} \caption{Summary statistics by attrition} \label{tab:sum-attr2} \maxsizebox*{\textwidth}{\textheight}{ \begin{threeparttable} \begin{tabular}{l*{3}{c}} \toprule
          &\multicolumn{3}{c}{Mean (SD)}\\\cmidrule(lr){2-4}
          &\multicolumn{1}{c}{Complete}&\multicolumn{1}{c}{Attrition}&\multicolumn{1}{c}{\specialcell{Complete -\\Attrition}}\\
\midrule
Monthly income&   112.86&    87.20&     0.29\\
          &(121.67) 284 &(103.58) 27 &         \\
Receives regular income&     0.11&     0.09&     0.84\\
          &(0.31) 145 &(0.30) 11 &         \\
Employed  &     0.51&     0.41&     0.31\\
          &(0.50) 284 &(0.50) 27 &         \\
Self-employed&     0.22&     0.18&     0.68\\
          &(0.42) 209 &(0.39) 22 &         \\
No. of dependants&     3.33&     3.07&     0.61\\
          &(2.49) 284 &(2.57) 27 &         \\
Subject is a dependant&     0.26&     0.15&     0.19\\
          &(0.44) 284 &(0.36) 27 &         \\
\bottomrule \end{tabular} \begin{tablenotes}[flushleft] \footnotesize \item \emph{Notes:} The first two columns report means of each row variable by observation status at endline. SD are in parentheses with sample size. The last column report the \emph{p}-value for a difference of means \emph{t}-test between each group. * denotes significance at 10 pct., ** at 5 pct., and *** at 1 pct. level. \end{tablenotes} \end{threeparttable} } \end{table}
