\begin{table}[h]\centering \def\sym#1{\ifmmode^{#1}\else\(^{#1}\)\fi} \caption{Multinomial treatment effects -- Gambling behavior} \label{tab:reg-mlogfreq} \maxsizebox*{\textwidth}{\textheight}{ \begin{threeparttable} \begin{tabular}{l*{4}{c}} \toprule
          &\multicolumn{3}{c}{Relative risk ratio}&\multicolumn{1}{c}{Sample}\\\cmidrule(lr){2-4}\cmidrule(lr){5-5}
          &\multicolumn{1}{c}{(1)}&\multicolumn{1}{c}{(2)}&\multicolumn{1}{c}{(3)}&\multicolumn{1}{c}{(4)}\\
          &\multicolumn{1}{c}{No Feedback}&\multicolumn{1}{c}{PLS}&\multicolumn{1}{c}{\specialcell{PLS-\\No Feedback}}&\multicolumn{1}{c}{Obs.}\\
\midrule
Gambled less&     0.91&     1.69&     1.86&      284\\
          &   (0.38)&   (0.66)&   (0.76)&         \\
Gambled more&     1.62&3.03\sym{***}&1.87\sym{*}&      284\\
          &   (0.69)&   (1.23)&   (0.69)&         \\
\bottomrule \end{tabular} \begin{tablenotes}[flushleft] \footnotesize \item \emph{Notes:} This table reports estimates from a multinomial logit estimation of the categorial response on treatment assigment. Each row corresponds to a response category with the baseline value as .  Columns 1--2 reports the treatment effect in relative risk ratios compared to the control group. Column 3 reports the difference between the two PLS treatments. Standard errors are in parentheses. Column 4 reports the number of observations in the analytic sample. Observations are at the individual level. * denotes significance at 10 pct., ** at 5 pct., and *** at 1 pct. level. \end{tablenotes} \end{threeparttable} } \end{table}

% File produced by reg-mlogit.do with /Users/justin/Repos/akiba-lottery-pub/data/clean/akiba_wide.dta on 13:18:28  5 May 2020 by user justin on Stata 13.1 with seed X27e0a1708256a41cdeaf4038df2ac2a9000400f4