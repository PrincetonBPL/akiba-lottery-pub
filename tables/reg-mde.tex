\begin{table}[h]\centering \def\sym#1{\ifmmode^{#1}\else\(^{#1}\)\fi} \caption{Minimum detectable effect sizes} \label{tab:reg-mde} \maxsizebox*{\textwidth}{\textheight}{ \begin{threeparttable} \begin{tabular}{l*{4}{c}} \toprule
          &\multicolumn{1}{c}{(1)}&\multicolumn{1}{c}{(2)}&\multicolumn{1}{c}{(3)}&\multicolumn{1}{c}{(4)}\\
          &\multicolumn{1}{c}{Lottery MDE}&\multicolumn{1}{c}{\specialcell{Control Mean\\(SD)}}&\multicolumn{1}{c}{N}&\multicolumn{1}{c}{col4}\\
\midrule
Total no. of deposits&     7.09&     7.98&    13.66&      311\\
          &         &         &  (15.08)&         \\
No. of days saved&     5.77&     6.52&    11.78&      311\\
          &         &         &  (12.93)&         \\
Daily avg. no. of deposits&     0.12&     0.13&     0.23&      311\\
          &         &         &   (0.25)&         \\
Total deposit amount&     9.38&     8.10&    14.87&      311\\
          &         &         &  (24.48)&         \\
Total withdrawal amount&     2.65&     2.87&     1.07&      311\\
          &         &         &   (4.53)&         \\
Total savings last month (USD PPP)&    70.69&    67.60&    80.31&      284\\
          &         &         & (112.74)&         \\
M-Pesa savings last month (USD PPP)&    17.80&    14.89&    20.42&      284\\
          &         &         &  (44.67)&         \\
ROSCA savings last month (USD PPP)&    18.98&    20.61&    22.24&      283\\
          &         &         &  (42.18)&         \\
Currently saves with ROSCA&     0.20&     0.20&     0.54&      284\\
          &         &         &   (0.50)&         \\
Gamble more&     0.14&     0.17&     0.12&      284\\
          &         &         &   (0.32)&         \\
\bottomrule \end{tabular} \begin{tablenotes}[flushleft] \footnotesize \item \emph{Notes:} Columns 1 and 2 report the minimum detectable effect sizes of the lottery treatment compared to control and the regret treatment against the lottery, respectively, with \(\alpha\) = 0.05 and 0.8 power. Columns 3--4 report the control group means and SDs and size of the analytic sample. \end{tablenotes} \end{threeparttable} } \end{table}
